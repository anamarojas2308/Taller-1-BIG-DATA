\documentclass[12pt,a4paper,onecolumn]{article}

%%%%%%%%%%%%%%%%%%%%%%%%%%%%%%%%%%%
%          				PACKAGES  				              %
%%%%%%%%%%%%%%%%%%%%%%%%%%%%%%%%%%%

\usepackage[margin=1in]{geometry}
\usepackage{authblk}
\usepackage[latin1]{inputenc}
\usepackage{amsfonts}
\usepackage{a4wide,graphicx,color}
\usepackage{amsmath}
\usepackage{amssymb}
\usepackage[table]{xcolor}
\usepackage{setspace}
\usepackage{booktabs}
\usepackage{dcolumn}
\usepackage{rotating}
\usepackage{color,soul}
\usepackage{threeparttable}
\usepackage[capposition=top]{floatrow}
\usepackage[labelsep=period]{caption}

\usepackage{subcaption}
\usepackage{lscape}
\usepackage{pdflscape}
\usepackage{multicol}
\usepackage[bottom]{footmisc}
\setlength\footnotemargin{5pt}
\usepackage{longtable} %for long tables

\usepackage{enumerate}
\usepackage{units}  %nicefraction
\usepackage{placeins}
\usepackage{booktabs,multirow}
%% BibTeX settings
\usepackage{natbib}
\bibliographystyle{apalike}
%\bibliographystyle{unsrtnat}
\bibpunct{(}{)}{,}{a}{,}{,}


%% paragraph formatting
\renewcommand{\baselinestretch}{1}


% Defines columns for tables
\usepackage{array}
\newcolumntype{L}[1]{>{\raggedright\let\newline\\\arraybackslash\hspace{0pt}}m{#1}}
\newcolumntype{C}[1]{>{\centering\let\newline\\\arraybackslash\hspace{0pt}}m{#1}}
\newcolumntype{R}[1]{>{\raggedleft\let\newline\\\arraybackslash\hspace{0pt}}m{#1}}

\usepackage{comment} %to comment entire sections

\usepackage{xfrac} %sideways fractions


\usepackage{bbold} %for indicators

\setcounter{secnumdepth}{6}  %To get paragraphs referenced 

\usepackage{titlesec} %subsection smaller
\titleformat*{\subsection}{\normalsize \bfseries} %subsection smaller
%\usepackage[raggedright]{titlesec} % for sections does not hyphen words


\usepackage[colorlinks=true,linkcolor=black,urlcolor=blue,citecolor=blue]{hyperref}  %Load last
%% markup commands for code/software
\let\code=\texttt
\let\pkg=\textbf
\let\proglang=\textsf
\newcommand{\file}[1]{`\code{#1}'}
\newcommand{\email}[1]{\href{mailto:#1}{\normalfont\texttt{#1}}}
\urlstyle{same}

%%%%%%%%%%%%%%%%%%%%%%%%%%%%%%%%%%%
%     			TITLE, AUTHORS AND DATE    			  %
%%%%%%%%%%%%%%%%%%%%%%%%%%%%%%%%%%%
%% Title, authors and date

\title{Problem Set 1: Prediciendo el Ingreso}

\author{Mario Andr\'es Mercado, Julian Delgado, Ana Mar\'ia, Juan David} 
\date{\today}
\begin{document}



\maketitle

\thispagestyle{empty} % Leaves first page without page number

%%%%%%%%%%%%%%%%%%%%%%%%%%%%%%%%%%%
%          			  ABSTRACT       					      %
%%%%%%%%%%%%%%%%%%%%%%%%%%%%%%%%%%%









% Keywords and JEL Classification
\medskip



% Ends title page and defines spacing for the rest of the document
\pagebreak
\doublespacing

%%%%%%%%%%%%%%%%%%%%%%%%%%%%%%%%%%%
%    DOCUMENT    		          %
%%%%%%%%%%%%%%%%%%%%%%%%%%%%%%%%%%%




\section{Introduction} \label{sec:intro}

In the public sector, accurate reporting of individual income is critical for computing taxes.
However, tax fraud of all kinds has always been a significant issue. According to the Internal
Revenue Service (IRS), about 83.6\% of taxes are paid voluntarily and on time in the US. \footnote{\url{https://www.irs.gov/newsroom/the-tax-gap}} One of the causes of this gap is the under-reporting of incomes by individuals. An income
predicting model could potentially assist in flagging cases of fraud that could lead to the
reduction of the gap. Furthermore, an income prediction model can help identify vulnerable
individuals and families that may need further assistance.\\
The objective of the problem set is to apply the concepts we learned using real world
data. For that, we are going to scrape from the following website: \url {https://ignaciomsarmiento.github.io/GEIH2018-sample/}. This website contains data for Bogot\'a from the 2018 Medici\'on de Pobreza Monetaria y Desigualdad Report that takes information from the \href{https://www.dane.gov.co/index.php/estadisticas-por-tema/mercado-laboral/empleo-y-desempleo/geih-historicos}{GEIH}

\subsection{General Instructions}

The main objective is to construct a model of individual hourly wages

\begin{equation}
    w = f(X)+u
\end{equation}

where $w$ is the hourly wage, and $X$ is a matrix that includes potential explanatory variables/predictors. In this problem set, we will focus on $f(X) = X\beta$.\\
The final document, in .pdf format, must contain the following sections:

\begin{enumerate}
    \item \textit{Introduction.} The introduction briefly states the problem and if there are any antecedents. It briefly describes the data and its suitability to address the problem set question. It contains a preview of the results and main takeaways.

    La subdeclaraci\'on de ingresos es un problema recurrente en la recaudaci\'on de impuestos, contribuyendo a la evasi\'on fiscal. Por lo cual, se considera necesario un modelo de predicci\'on de ingresos para detectar fraudes y asistir a personas vulnerables que necesiten focalizaci\'on y desarrollo de pol\'iticas p\'ublicas. Por un lado, el nivel de ingresos en el mercado laboral es clave en la econom\'ia y en la formulaci\'on de pol\'iticas p\'ublicas, ya que influye en la distribuci\'on de la riqueza y en la calidad de vida de la poblaci\'on. Diversos estudios han analizado c\'omo factores como la edad, el g\'enero, el nivel educativo y el estrato socioecon\'omico afectan los ingresos de los trabajadores. Este trabajo busca explorar la relaci\'on entre estas variables y los salarios, con un \'enfasis particular en la estructura de los ingresos por hora. Para ello, se utiliza una amplia base de datos representativa de trabajadores en Bogot\'a, la cual permite examinar patrones de ingreso en funci\'on de la edad, el g\'enero, la educaci\'on, el estrato socioecon\'omico y la formalidad. Se emplean diversas herramientas estad\'isticas y visualizaciones para identificar tendencias y disparidades en los ingresos, y en particular, se modela el salario por hora en t\'erminos logar\'itmicos para evaluar la relaci\'on no lineal con la edad, considerando efectos decrecientes del salario a medida que aumenta la experiencia laboral. El estudio confirma la importancia de la educaci\'on y la experiencia en la determinaci\'on de los salarios y resalta las desigualdades estructurales existents en el mercado laboral. Estos hallazgos tienen implicaciones importantes para el dise\~no de pol\'iticas p\'ublicas orientadas a la reducci\'on de la desigualdad y la mejora de las condiciones laborales. Por el otro lado, los modelos estad\'isticos buscan estimar cu\'anto deber\'ia ganar un individuo en funci\'on de variables como edad, educaci\'on, tipo de empleo y sector econ\'omico. Si el ingreso declarado por una persona es significativamente menor al valor predicho por el modelo, podr\'ia ser un indicio de subdeclaraci\'on de ingresos. De este modo, los contribuyentes que reportan ingresos inusualmente bajos dentro de su grupo de referencia pueden ser marcados para auditor\'ia (outliers). Con ayuda de modelos de predicci\'on, an\'alisis de outliers, comparaci\'on con bases de datos externas y modelos de machine learning se logra detectar el fraude fiscal.

    \item \textit{Data.}\footnote{This section is located here so the reader can understand your work, but probably it should be the last section you write. Why? Because you are going to make data choices in the estimated models. And all variables included in these models should be described here} We will use data for Bogot\'a from the 2018 Medici\'on de Pobreza Monetaria y Desigualdad Report that takes information from the \href{https://www.dane.gov.co/index.php/estadisticas-por-tema/mercado-laboral/empleo-y-desempleo/geih-historicos}{GEIH}
    The data set contains all individuals sampled in Bogota and is available at the following website \url {https://ignaciomsarmiento.github.io/GEIH2018-sample/}. To obtain the data, you must scrape the website. In this problem set, we will focus only on employed individuals older than eighteen (18) years old. Restrict the data to these individuals and perform a descriptive analysis of the variables used in the problem set. Keep in mind that in the data, there are many observations with missing data or 0 wages. I leave it to you to find a way to handle this data.
    When writing this section up, you must:

\begin{enumerate}
    \item Describe the data briefly, including its purpose, and any other relevant information. \\
    
    Una vez se combinan los datos y se obtiene la base completa, se evidencia que hay informaci\'on que nos permite analizar el mercado laboral y variables ec\'onomicas de diferentes individuos. Se encuentra que el objetivo principal de la base de datos es analizar ingresos (salario) y comportamientos del mercado. Hay variables que nos permiten analizar caracter\'isticas individuales como la edad, g\'enero, educaci\'on, departamento, entre otras y variables que se relacionan con la actividad laboral como los ingresos, horas trabajadas, tipo de empleo, sector econ\'omico, entre otras. La selecci\'on de las variables es clave puesto que permitir\'a evaluar los factores que influyen en el ingreso y analizar fen\'omenos como la brecha salarial de g\'enero. De igual forma, es importante mencionar que la base tiene valores faltantes y observaciones con ingreso igual a cero, lo que sugiere la necesidad de un proceso de limpieza antes del an\'alisis. La base tiene una estructura de formato tabular, la cual facilita el procesamiento de los datos y el an\'alisis de los mismos mediante descripciones estad\'isticas. Adicionalmente, se considera que los datos proporcionados son fundamentales para construir un modelo predictivo de ingresos por hora y detectar como variables sociodemogr\'aficas y laborales inciden en la remuneraci\'on de los individuos.

    \item Describe the process of acquiring the data and if there are any restrictions to accessing/scraping these data. \\
    
    En primer lugar hay que tener presente que los datos no se pueden descargar como un archivo, sino que se encuentran en un p\'agina web \url{https://ignaciomsarmiento.github.io/GEIH2018_sample/} por lo cual, es necesario hacer web scrapping para extraerlos. Los datos estaban distribuidos en diferentes sitios web, por lo se combinaron en un \'unico conjunto de datos. De esta forma, se difini\'o la base de la URL, donde cada p\'agina se nombr\'o siguiendo un patr\'on (geih\_page\_1.html, geih\_page\_2.html ..., etc). Con ayuda de la libreria rvest se ley\'o el contenido HTML de cada p\'agina (data chunk 1:10), se extrajeron las tablas y se combinaron los datos en un solo dataframe, almacenando el resultado en la variables datos\_totales. Para evitar que el c\'odigo fuese a generar error en caso de que una de las p\'aginas no cargara correctamente, se utiliz\'o tryCatch (manejo de errores). De esta forma, se buscaba que si las p\'aginas cargaban correctamente, se extrayera las tablas y se a\~nadieran a la variable datos\_totales y si se llegaba a presentar alg\'un problema de lectura, se captar\'ia el error sin detener la ejecuci\'on total. Por lo tanto, el c\'odigo permite automatizar el proceso de recolecci\'on de datos que est\'an distribuidores en diferentes sitios web y cuando se trabaja con grandes vol\'umenes de datos, a\'un si se presenta alg\'un problema. No se encuentran restricciones dado que la p\'agina no tiene capchas ni bloqueos para la extracci\'on de datos, pero tuvimos presente no hacer m\'ultiples solicitudes en un corto per\'iodo de tiempo para no sobrecargar el servidor. Por el otro lado, los datos son p\'ublicos y fueron dise\~nados para fines acad\'emicos, por lo que facilito en gran medida el scrapping.




    
    \item Describe the data cleaning process and \\
    
    En primer lugar, se consider\'o necesario renombrar ciertas variables como edad, sexo, ingreso total para facilidad de lectura del grupo y se cre\'o una variable ln\_Salario\_hora, a la cual se le aplic\'o logar\'itmo natural para facilitar el an\'alisis de regresi\'on posterior. De acuerdo al enunciado del problem set, se identific\'o que deb\'iamos filtrar la base por aquellos individuos empleados mayores de 18 a\~nos. Por lo cual, se filtraron los datos para las muestras mayores a 18 a\~nos, donde el ingreso total fuese mayor a cero y tanto las horas trabajadas como el salario por hora fuese mayor a cero. Adicionalmente, se identificaron y eliminaron outliers extremos para las variables de horas trabajadas y salario por hora usando percentiles 1\% y 99\% con el fin de que dichos valores at\'ipicos distorsionen las interpretaciones de los resultados. Posteriormente, se seleccionaron \'unicamente variables n\'umericas y se elimnaron aquellas con desviaci\'on est\'andar igual a cero, puesto que no aporta variabilidad al modelo. Asimismo, las variables NA dado que no tienen informaci\'on \'util. 

Teniendo en cuenta que la base completa era bastante amplia, se decidi\'o calcular la correlaci\'on de cada variable num\'erica con Salario\_hora y de esta forma, se seleccionaron las 25 variables con mayor correlaci\'on puesto que ser\'ian las m\'as relevantes para analizar. Adicionalmente, se consideraron otras variables como la edad, el g\'enero, etc  dado que estas variables han sido ampliamente estudiadas en investigaciones econ\'omicas y del mercado laboral, demostrando efectos significativos en los ingresos. De este modo, las variables relevantes para el an\'alisis son de caracter\'isticas individuales (edad, genero, educaci\'on (college, maxEducLeval)), caracter\'isticas laborales (oficio, informalidad, horas trabajadas), caracter\'isticas econ\'omicas (ingresos, estrato) y as\'i se crea un nuevo dataframe con solo estas variables para reducir la dimensionalidad de la base. 

    \item Descriptive the variables included in your analysis. At a minimum, you should include a descriptive statistics table with its interpretation. However, I expect a deep analysis that helps the reader understand the data, its variation, and the justification for your data choices. Use your professional knowledge to add value to this section. Do not present it as a dry list of ingredients.\\
    
La tabla \ref{tab:descr} corresponde a las estad\'isticas descriptivas de las variables seleccionadas aquellas con una correlaci\'on significativa y aquellas ampliamente estudiadas). En primer lugar, se analizan las variables demogr\'aficas. Se evidencia que los individuos est\'an entre los 19 y los 86 a\~nos de edad, con una media de 36.3 a\~nos lo que sugiere que la muestra poblacional est\'a en edad laboral. La media de la variable de g\'enero (hombre) es de 0.5038, lo que indica que hay una distribuci\'on equilibrada entre hombres y mujeres. A partir de la variable  maxEducLevel se evidencia que la mayor\'ia de la poblaci\'on tiene al menos educaci\'on secundaria o t\'ecnica, dado que la media es 6 y a partir de college se evidencia que el alrededor del 35\% de la muestra ha alcanzado la educaci\'on superior (universitaria), dado que la media es de 0.3474. De esta forma, se puede concluir que la muestra incluye una poblaci\'on trabajadora en su mayor\'ia joven-adulta, con igualdad de representaci\'on entre hombres y mujeres y un nivel educativo medio-alto. 

En segundo lugar, se analizan las variables laborales y se encuentra a partir de la variable informal que el 21.41 \% de la muestra est\'a en el sector informal (informal = 0.2141), mientras que el 78.59\% est\'a en el sector formal (formal = 0.7859). Adicionalmente, se encuentra por medio de la variable hoursWorkUsual que la mayor\'ia trabaja jornadas laborales completas puesto que la media es de 48.13 horas y que no hay individuos independientes dado que la media de la variable cuenta propia es cero. 

En tercer lugar, se analizan las variables econ\'omicas y se encuentra la variable de Salario por hora, la cual tiene un rango de 1,215 a 58,333, con una mediana de 4,555 y una media de 7,291.La gran diferencia entre media y mediana sugiere que hay valores extremos que elevan el promedio. La variable de ingreso total(ingtotes) muestra una gran dispersi\'on en los valores, con un m\'aximo de 8 millones. La media (434,783) es significativamente mayor a la mediana (191,834), indicando la presencia de ingresos muy elevados en la parte alta de la distribuci\'on y, por medio de Logaritmo del salario (ln\_Salario\_hora) se evidencia que los ingresos siguen una distribuci\'on normal. Es decir, los ingresos presentan una fuerte dispersi\'on y una distribuci\'on sesgada con valores muy altos en la parte superior. Esto refuerza la necesidad de usar el logaritmo del salario en los modelos de regresi\'on para reducir heterocedasticidad. 

Finalmente, se analizan las variables sociales y se encuentra a partir de la variable de estrato socioecon \'omico que la poblaci\'on est\'a en estratos medios-bajos pues la media es de 2.491. Sin embargo, se encuentra un m\'aximo de 6, lo que refleja la diversidad de condiciones socioecon\'omicas en la muestra. Por medio de la variable (cotPension), se evidencia que la mayor\'ia de los trabajadores est\'an afiliados a pensi\'on pues la media de 1.232 indica que m\'as del 50\% cotiza. De esta forma, el acceso a la seguridad social est\'a presente en una parte importante de la poblaci\'on, pero es posible que una proporci\'on significativa no cotice. La distribuci\'on de estratos refleja la heterogeneidad econ\'omica en Bogot\'a. En conclusi\'on, la poblaci\'on trabajadora se compone principalmente por adultos j\'ovenes con una distribuci\'on casi equitativa entre hombres y mujeres. La mayor\'ia de los trabajadores est\'an en el sector formal y con jornadas laborales completas, aunque hay presencia de informalidad y posibles casos de sobreempleo. Los ingresos presentan una alta dispersi\'on, con valores extremos que sesgan la media hacia valores elevados.Existe una diversidad en niveles educativos y estratos socioecon\'omicos, lo que puede ser clave en la modelizaci\'on del salario.

Para completar el an\'alisis de las estad\'isticas descriptivas, se consider\'o pertinente ver la relaci\'on gr\'afica entre la variable salario y otras variables de inter\'es como las mencionadas anteriormente. La primera relaci\'on que encontramos es la de la Figura 1. Age-Earnings Profile, en el que se muestra la relaci\'on entre la edad y el salario por hora estimado. Se observa una curva en forma de U c\'oncava, lo que sugiere que los ingresos tienden a aumentar con la edad hasta un punto m\'aximo alrededor de los 45-50 a\~nos y luego disminuyen progresivamente. Esto puede explicarse por la acumulaci\'on de experiencia y educaci\'on en la primera mitad de la vida laboral, seguida por una posible reducci\'on en la intensidad del trabajo, jubilaci\'on o cambios en el tipo de empleo en una edad mayor. 

En la Figura \ref{fig:rplot1} Ingresos promedio por grupo de edad y g\'enero, se evidencian los ingresos promedio por grupo de edad y se diferencia entre hombres (azul) y mujeres (rojo). Se analiza que en la mayor\'ia de los grupos de edad, los hombres tienden a tener ingresos m\'as altos que las mujeres, aunque en algunos rangos (como 40-50 a\~nos) la diferencia es menor. Hay una tendencia similar al primer gr\'afico, donde los ingresos crecen con la edad hasta un punto m\'aximo entre 40-60 a\~nos y luego disminuyen en el grupo de mayores de 70 a\~nos. El grupo de menores de 20 a\~nos, tiene los ingresos m\'as bajos, lo cual tiene sentido puesto que est\'an en etapas de educaci\'on o comenzando la carrera laboral. Ambos gr\'aficos reflejan una relaci\'on clara entre la edad y los ingresos, con un punto m\'aximo alrededor de los 40-50 a\~nos y una posterior disminuci\'on. Tambi\'en se observa una brecha de g\'enero en los ingresos, con los hombres ganando m\'as en la mayor\'ia de las edades. Esto puede estar influenciado por diferencias en acceso a empleos mejor remunerados, interrupciones laborales en el caso de las mujeres (embarazo) y otros factores estructurales del mercado laboral. 

En la Figura \ref{fig:rplot2} Ingresos promedio por estrato se muestra la relaci\'on entre el estrato socioecon\'omico y el ingreso promedio de la muestra poblacional. Se observa una relaci\'on positiva entre estrato e ingresos, puesto que a medida que el estrato aumenta, el ingreso promedio tambi\'en crece. Esto es posible, ya que los estratos m\'as altos suelen tener acceso a mejores oportunidades econ\'omicas y educativas. Sin embargo, se evidencia una gran diferencia entre el estrato 1 (el m\'as bajo) y los estratos superiores. Por ejemplo, el estrato 1 tiene ingresos muy bajos en comparaci\'on con el estrato 2, que experimenta un fuerte incremento.Los estratos 5 y 6 tienen ingresos similares, por lo que no se observa un aumento significativo entre estos dos estrato. Esto indica que, a partir de cierto punto, el crecimiento del ingreso se estabiliza.Este gr\'afico refleja la desigualdad econ\'omica y c\'omo el acceso a recursos (vivienda, educaci\'on) influye en los ingresos. 

La Figura \ref{fig:rplot3} Ingresos promedio por nivel de educaci\'on muestra c\'omo el nivel educativo impacta en el ingreso promedio, diferenciando entre hombres (azul) y mujeres (rojo).Se evidencia una clara relaci\'on positiva entre el nivel educativo y el ingreso promedio. Aquellos con mayor nivel de estudios obtienen ingresos significativamente m\'as altos. Asimismo, se evidencia que en los niveles educativos m\'as bajos, las diferencias salariales entre hombres y mujeres se perciben, mientras que a medida que aumenta el nivel educativo, las diferencias de g\'enero se reducen, especialmente en los niveles m\'as altos donde los ingresos parecen igualarse. El salto en ingresos entre los niveles educativos bajos y altos es notable. La diferencia entre los que tienen poca educaci\'on y los que alcanzan niveles universitarios es muy marcada.Este gr\'afico refuerza la importancia de la educaci\'on como un determinante clave de los ingresos y sugiere que una mayor escolaridad puede ser una vía para mejorar las condiciones econ\'omicas.

En la Figura \ref{fig:rplot6} Salario-hora por horas de trabajo se muestra la relaci\'on entre las horas de trabajo (eje X) y el salario por hora (eje Y), diferenciando entre trabajadores formales (azul) e informales (rojo).Principalmente, se observa que los trabajadores formales tienden a tener salarios por hora m\'as altos, con una mayor dispersi\'on de valores en la parte superior del gr\'afico, mientras que los trabajadores informales se concentran en los niveles m\'as bajos de salario por hora, con pocos casos de ingresos altos. Asimismo, se observa que la mayor\'ia de los trabajadores se agrupan entre 30 y 60 horas semanales. No hay una clara tendencia de aumento del salario por hora con m\'as horas trabajadas, lo que indica que la cantidad de horas no garantiza mejores ingresos. Finalmente, se observa que hay trabajadores informales con techo de ingresos dado que casi todos los puntos rojos est\'an en la parte baja del gr\'afico, lo que indica que presentan limitaciones salariales y no alcanzan los mismo niveles salariales que los trabajadores formales.Por \'ultimo, se analiza 

la Figura \ref{fig:rplot9} Boxplot promedio de ingresos por estrato, la cual muestra la distribuci\'on de los ingresos promedio en funci\'on del estrato socioecon\'omico. Se encuentra una tendencia creciente de ingresos con respecto al estrato. Los estratos m\'as bajos (1, 2 y 3) tienen ingresos considerablemente menores y con menor dispersi\'on, mientras que los estratos altos (4, 5 y 6) presentan ingresos m\'as elevados y una mayor variabilidad. Por otro lado, se encuentra que en los estratos 1, 2 y 3, los ingresos est\'an muy concentrados en valores bajos, con algunas excepciones representadas por los outliers. A partir del estrato 4, la mediana de los ingresos aumenta notablemente y la dispersi\'on de los datos es mucho mayor. El estrato 6 tiene la distribuci\'on m\'as amplia, lo que indica una mayor diferencia entre quienes ganan poco y quienes ganan mucho dentro de este grupo. Es importante mencionar que se observan outliers en todos los estratos, pero en los m\'as bajos estos outliers son menos frecuentes y no tan extremos, mientras que en los estratos m\'as altos, la presencia de valores at\'ipicos con ingresos significativamente altos sugiere que hay grupos reducidos de personas con ingresos muy elevados. 

De esta manera, se puede concluir que hay una fuerte relaci\'on entre variables socioecon\'omicas y los ingresos de la poblaci\'on. Se observa que el nivel de ingresos aumenta con la edad hasta cierto punto, mostrando un pico, con diferencias notables entre hombres y mujeres. Asimismo, el estrato socioecon\'omico y el nivel educativo tienen un impacto significativo en los ingresos, donde los estratos altos y las personas con mayor nivel educativo perciben salarios más elevados. Adem\'as, el mercado laboral presenta disparidades entre el empleo formal e informal, con trabajadores formales obteniendo mayores salarios por hora, aunque con una variabilidad considerable. Finalmente, la distribuci\'on de ingresos refleja altos niveles de desigualdad, especialmente en los estratos m\'as altos, donde hay una mayor dispersi\'on y presencia de outliers con ingresos significativamente altos. Estos resultados demuestran la importancia de desarrollar políticas p\'ublicas que promuevan el acceso a la educaci\'on y la formalizaci\'on del empleo como estrategias clave para reducir las brechas econ\'omicas y laborales.







    

\end{enumerate}

 
    \item \textit{Age-wage profile.} A great deal of evidence in labor economics suggests that the typical workers age-wage profile has a predictable path: Wages tend to be low when the worker is young; they rise as the worker ages, peaking at about age 50; and the wage rate tends to remain stable or decline slightly after age 50

    In this subsection we are going to estimate the Age-wage profile profile for the individuals in this sample:

    \begin{equation}
    log(w_{i}) = \beta_{1} +\beta_{2}Age_{i} + \beta_{3}Age_{i}^{2} + u
    \end{equation}

    \begin{enumerate}
    \item Regression table
    
     Ver \ref{tab:wage1} 
    
    \item Interpretation\\
    $\beta_{1}$ $Intercept$. Represents the baseline value of log(wage) when Age is 0.
    $\beta_{2}$ $Age$ Coefficient. Represents the change in log(wage) for each additional year of age.
    $\beta_{3}$ $Age^{2}$ Coefficient. Captures the non-linear effect of age on wages, accounting for the curvilinear relationship.


    
    \item Discussion \\
    La primera tabla muestra la estimaci\'on de un modelo de regresi\'on para explicar el logaritmo del salario por hora (ln\_salario\_hora) en funci\'on de la edad y la edad al cuadrado. En primer lugar, se analiza el coeficiente edad, donde edad = 0.054, p<0.01. Se encuentra que por cada a\~no adicional de edad, el salario por hora aumenta 5,4\%, manteniendo constantes las dem\'as variables. El error est\'andar es peque\~no (0.003), lo que sugiere que la estimaci\'on es precisa. Dado que el coeficiente es estad\'isticamente significativo a un nivel del 1\% (p<0.01), hay fuerte evidencia de que la edad tiene un impacto positivo en los salarios por hora. En segundo lugar, se analiza la edad al cuadrado, donde edad\_squared = -0.001, p<0.01. El coeficiente negativo sugiere que, a medida que aumenta la edad, el crecimiento del salario por hora se desacelera. Es decir, la relaci\'on entre edad e ingresos sigue una curva cuadr\'atica (c\'oncava), lo que indica que hay un punto m\'aximo a partir del cual los salarios dejan de crecer y comienzan a disminuir. Tambi\'en es significativo a un nivel del 1\% (p<0.01), lo que confirma que esta relaci\'on cuadr\'atica es relevante en el modelo. La constante igual a 7.527 representa el valor de ln\_salario\_hora cuando edad = 0. Su valor elevado sugiere que, en t\'erminos logar\'itmicos, los salarios de referencia sin considerar la edad son relativamente altos. 

Por otro lado, se considera necesario analizar los indicadores estad\'isticos del modelo. Por ejemplo, observamos que la muestra usada en la estimaci\'on del modelo incluye 9423 individuos. El $R^{2}$ indica que el modelo solo explica el 3.5\% (0.035) de la variabilidad en los salarios por hora, lo cual nos indica que hay otros factores importantes que determinan el salario y que no est\'an incluidos en el modelo como educaci\'on, experiencia o sector de empleo. El error est\'andar residual (0.634) muestra la dispersi\'on de los errores del modelo, indicando qu\'e tan lejos est\'an los valores reales de los valores predichos en promedio. Finalmente, se logra concluir que el modelo es globalmente significativo dado que valor p del Estad\'istico F es menos a 0.01 se logra rechazar la hip\'otesis nula de que todos los coeficientes son iguales a cero. 

Siendo as\'i, el modelo sugiere que el salario por hora aumenta con la edad, pero a un ritmo decreciente, alcanzando un punto m\'aximo antes de empezar a disminuir. Sin embargo, el bajo $R^{2}$ indica que la edad por s\' sola no es un buen predictor del salario y que se necesitan otras variables (como nivel educativo, experiencia, g\'enero, sector laboral, etc.) para explicar mejor la variabilidad en los ingresos.

    \item plot of the estimated age-earnings

    ver figura  \ref{fig:rplot1}
    
    \end{enumerate}































    \item \textit{The gender earnings GAP.} Policymakers have long been concerned with the gender wage gap, and is going to be our focus in this subsection.

    \begin{enumerate}
    \item Begin by estimating and discussing the unconditional wage gap:
    
    \begin{equation}
    log(w_{i}) = \beta_{1} +\beta_{2}Female_{i}  + u
    \end{equation}

    where F emale is an indicator that takes one if the individual in the sample is identified as female.

     Vease table (\ref{tab:wagefemale}) 



    
    \item \textit{Equal Pay for Equal Work?} A common slogan is equal pay for equal work. One way to interpret this is that for employees with similar worker and job characteristics, no gender wage gap should exist. Estimate a conditional earnings gap incorporating control variables such as similar worker and job characteristics. In this section, estimate the conditional wage gap:
    First, using FWL
    Second, using FWL with bootstrap. Compare the estimates and the standard errors.
    

  \begin{equation}
    log(w_{i}) = \beta_{1} +\beta_{2}Female_{i} + X\gamma + u 
    \end{equation}



    \item Next, plot the predicted age-wage profile and estimate the implied peak ages with the respective confidence intervals by gender.

    Vease figura (\ref{fig:female}) 
    
     Vease tabla (\ref{tab:wagecontrol}) 

     Las variables de control son escolaridad, edad, $edad^{2}$, cuentapropia, jefe de hogar, formal y ocupado. Las variables explicativas que pueden ser malos controles, son formal, ocupado o cuentapropista, la explicaci\'on se debe a que estas variables pueden estar correlacionas entre ellas, al igual que con la variable predictora (logaritmo de salarios), es el caso de ser formal, esta posiblemente est\'e fuertemente relacionada con la variable dependiente, o lo que es lo mismo presentar problemas de endogeneidad (econometr\'ia cl\'asica).

     Interpretaci\'on de los coeficientes:

     Para el modelo de brecha salarial incondicional, observamos que el coeficiente es de -0.028 y es estadisticamente significativo al 95\% de confianza. Este quiere decir que la mujer gana en promedio 2.8 pesos menos con respecto al hombre, por lo que la brecha salarial es considerable. Si se incluyen las variables de control, este valor se incrementa y llega a -0.067 es estadisticamente significativo al 99\% de confianza. En este caso, la brecha se incrementa un poco y ahora la mujer gana en promedio 6.7 menos pesos que el hombre evidenciando así la discriminaci\'on en el mercado de trabajo.

    
    

    \end{enumerate}
    
    \item \textit{Predicting earnings.} In the previous sections, you estimated some specifications with inference in mind. In this subsection, we will evaluate the predictive power of these specifications.

\begin{enumerate}
    \item Split the sample into two: a training (70\%) and a testing (30\%) sample. (Dont forget to set a seed to achieve reproducibility. In R, for example you can use set.seed(10101), where 10101 is the seed.)
    \item Report and compare the predictive performance in terms of the RMSE of all the previous specifications with at least five (5) additional specifications that explore non-linearities and complexity.

   ver figuras (\ref{tab:5}, \ref{tab:5.1} y \ref{tab:5.2})



    
    \item In your discussion of the results, comment:
    About the overall performance of the models.
    About the specification with the lowest prediction error.
    For the specification with the lowest prediction error, explore those observations that seem to miss the mark. To do so, compute the prediction errors in the test sample, and examine its distribution. Are there any observations in the tails of the prediction error distribution? Are these outliers potential people that the DIAN should look into, or are they just the product of a flawed model?


    \begin{itemize}
        \item Desempe\~no de los modelos: En general, los modelos presentan un desempe\~no relativamente bueno, con RMSE que oscilan entre 0.626 y 0.634. Dado que ahora solo usamos las variables edad y g\'enero (mujer), la capacidad predictiva ha sido m\'as conservadora en comparaci\'on con los modelos anteriores, lo que sugiere que la eliminaci\'on de variables como escolaridad ha reducido la complejidad del modelo, pero tambi\'en su precisi\'on.
        \item Modelos con mejor desempe\~no: El modelo con el menor RMSE es el Modelo 2.5 (0.626), seguido muy de cerca por el Modelo 2.3 (0.627). Estos modelos incluyen combinaciones no lineales de edad (como su cuadrado) e interacciones con g\'enero. Esto reafirma que, aunque las variables predictivas son limitadas, la inclusi\'on de t\'erminos no lineales sigue aportando valor, al capturar relaciones complejas entre edad y salario.
        \item Distribuci\'on de errores: La distribuci\'on de los errores de predicci\'on muestra una forma aproximadamente normal, con una fuerte concentraci\'on alrededor de 0. Esto indica que, en general, los modelos tienden a predecir los salarios con una precisi\'on razonable, sin sesgo sistem\'atico claro.
        \item Asimetr\'ia positiva: Hay una asimetr\'ia positiva evidente, con valores m\'as alejados hacia la derecha. Esto sugiere que algunos salarios fueron significativamente subestimados por los modelos, lo cual podr\'ia reflejar que existen individuos con ingresos muy altos que las variables edad y g\'enero no logran explicar adecuadamente.
        \item Outliers: El an\'alisis de outliers (percentiles 1\% y 99\%) revela que hay observaciones extremas con errores considerables. Estos outliers podr\'ian estar relacionados con personas que reciben salarios inusualmente bajos o altos, potencialmente debido a factores no contemplados por los modelos, como la ocupaci\'on, el nivel educativo o experiencia laboral.
        
    \end{itemize}







    
    \item LOOCV. For the two models with the lowest predictive error in the previous section, calculate the predictive error using Leave-one-out-cross-validation (LOOCV). Compare the results of the test error with those obtained with the validation set approach and explore the potential links with the influence statistic. (Note: when attempting this subsection, the calculations can take a long time, depending on your coding skills, plan accordingly!)\\

    Tras el an\'alisis realizado, se obtuvieron los siguientes resultados en cuanto al desempe\~no predictivo de los modelos estimados:
    
    Desempe\~no en el conjunto de validaci\'on:
    El Modelo 2.5 present\'o el menor RMSE, con un valor de 0.626.
    El Modelo 2.3 le sigui\'o de cerca, con un RMSE de 0.627.
    Desempe\~no con validaci\'on cruzada (LOOCV):
    El RMSE LOOCV para el Modelo 2.5 fue de 0.6331.
    El RMSE LOOCV para el Modelo 2.3 fue de 0.6340.
    
    Ambos modelos incluyen combinaciones no lineales de la variable edad, como su t\'ermino cuadr\'atico, y consideran las interacciones con el g\'enero (mujer), lo que permite capturar relaciones complejas entre estas variables y el salario por hora.
    A pesar de que el Modelo 2.5 obtuvo el menor RMSE tanto en validaci\'on como en LOOCV, la diferencia con el Modelo 2.3 es marginal. Esto sugiere que ambos modelos tienen un desempe\~no muy similar y que las mejoras adicionales logradas por el Modelo 2.5 son leves, aunque consistentes.
    El hecho de que las puntuaciones de RMSE en validaci\'on y LOOCV no difieran significativamente indica que los modelos no est\'an sobreajustados y que su capacidad predictiva es relativamente estable al enfrentarse a nuevos datos.
    
    Sin embargo, es importante destacar que, dado que solo se utilizaron las variables edad y g\'enero, las predicciones a\'un presentan limitaciones. La presencia de outliers y la asimetr\'ia positiva en los errores sugieren que hay factores adicionales como la escolaridad, la ocupaci\'on o la experiencia laboral que no fueron considerados y que podr\'ian ayudar a mejorar el poder explicativo de los modelos.
    
    En conclusi\'on, el Modelo 2.5 se posiciona como el mejor modelo predictivo dentro de los estimados, aunque el Modelo 2.3 ofrece resultados muy similares. Para futuras investigaciones, se recomienda explorar la inclusi\'on de nuevas variables y el uso de t\'ecnicas m\'as avanzadas para robustecer las predicciones.





    
\end{enumerate}

    
\end{enumerate}



%%%%%%%%%%%%%%%%%%%%%%%%%%%%%%%%%%%
%		  TABLES				  %
%%%%%%%%%%%%%%%%%%%%%%%%%%%%%%%%%%%
\section*{Tables and Figures}

\input{tables/tabledescrp}%
\input{tables/table3}%
\input{tables/table4}%
\input{tables/table4.1}%
\input{tables/table5.1}%
\input{tables/table5.2}%
\input{tables/table5}

\pagebreak

\begin{figure}[H]
\caption{Age-Earnings Profile} \label{fig:robust}
    \includegraphics[scale=0.75]{figures/Rplot.png}   
 \flushleft
\end{figure}


\begin{figure}[H]
\caption{Ingresos promedio por grupo de edad} \label{fig:rplot1}
    \includegraphics[scale=0.75]{figures/Rplot01.png}   
 \flushleft
\end{figure}


\begin{figure}[H]
\caption{Ingresos promedio por estrato} \label{fig:rplot2}
    \includegraphics[scale=0.75]{figures/Rplot02.png}   
 \flushleft
\end{figure}



\begin{figure}[H]
\caption{Ingresos promedio por nivel de educaci\'on} \label{fig:rplot3}
    \includegraphics[scale=0.75]{figures/Rplot03.png}   
 \flushleft
\end{figure}


\begin{figure}[H]
\caption{Ingresos promedio por oficio} \label{fig:rplot4}
    \includegraphics[scale=0.75]{figures/Rplot04.png}   
 \flushleft
\end{figure}


\begin{figure}[H]
\caption{Salario-hora por horas de trabajo} \label{fig:rplot6}
    \includegraphics[scale=0.75]{figures/Rplot06.png}   
 \flushleft
\end{figure}



\begin{figure}[H]
\caption{Salario-hora por oficio} \label{fig:rplot7}
    \includegraphics[scale=0.75]{figures/Rplot07.png}   
 \flushleft
\end{figure}



\begin{figure}[H]
\caption{Salario-hora por estrato} \label{fig:rplot8}
    \includegraphics[scale=0.75]{figures/Rplot08.png}   
 \flushleft
\end{figure}



\begin{figure}[H]
\caption{Boxplot promedio de ingresos por estrato} \label{fig:rplot9}
    \includegraphics[scale=0.75]{figures/Rplot09.png}   
 \flushleft
\end{figure}


\begin{figure}[H]
\caption{Distribuci\'on de errores de predici\'on} \label{fig:5}
    \includegraphics[scale=0.75]{figures/5.png}   
 \flushleft
\end{figure}


\begin{figure}[H]
\caption{Outliers edad-salario (log)} \label{fig:5.1}
    \includegraphics[scale=0.75]{figures/5.1.png}   
 \flushleft
\end{figure}

\begin{figure}[H]
\caption{Predicted age-wage profile} \label{fig:female}
    \includegraphics[scale=0.75]{figures/female.png}   
 \flushleft.
\end{figure}


%%%%%%%%%%%%%%%%%%%%%%%%%%%%%%%%
%   APPENDIX	 Tables	        %
%%%%%%%%%%%%%%%%%%%%%%%%%%%%%%%%
\pagebreak
\appendix
\renewcommand{\theequation}{\Alph{chapter}.\arabic{equation}}

\setcounter{figure}{0}
\setcounter{table}{0}
\makeatletter 
\renewcommand{\thefigure}{A.\@arabic\c@figure}
\renewcommand{\thetable}{A.\@arabic\c@table}

\end{document}
